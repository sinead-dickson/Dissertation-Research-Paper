\section{Introduction}
\IEEEPARstart{T}{he rapid} development and expansion of the internet has introduced new ways for individuals to express their opinions and disseminate their views. Online reviews have become a hugely important source of information for consumers. They play an important role in determining whether a person is satisfied with a product or service. Online reviews provide a huge amount of data on consumer preferences. It is now relatively rare that someone will purchase a product, reserve a table at a restaurant or book a room in a hotel without first checking some online reviews. These reviews inform and influence consumer decisions, and have a direct relationship with online sales. 

This project will focus on Twitter, a microblogging social media site, where users can post short blocks of text of no more than 280 characters. Twitter currently has ~126 million people who use the site daily, which the company terms `monetizable Daily Active Users' (mDAU), with over 500 million tweets posted per day \cite{Twitter2019}. These tweets can often take the form of a review.

Twitter users post tweets which express their ideas and opinions on a wide array of topics. An individual may, for example, tweet about a city they visited, a hotel they stayed in, a restaurant they ate at or a film they watched. These review-like tweets can give insight into consumers' opinions on the entities with which they interact. With millions of tweets being posted every day, Twitter has a huge potential source of underutilised reviews.

Traditional online review sites include websites like TripAdvisor, Foursquare and Yelp. Often these sites have a dedicated area for feedback and encourage users to leave reviews of their hotels, restaurants or products. In our opinion, this method of obtaining reviews can sometimes result in more forced, manufactured, less reliable reviews. It is our contention that users tend to post their genuine feelings more spontaneously and frequently to Twitter. Users often wouldn't consider their posts to be a 'review' in the formal sense of the word and not in the same way they would consider a review left in the dedicated feedback area of another website.

The language used in tweets is different to longer form text and needs to be treated differently. Tweets are usually very informal, using casual language and slang. They contain things like hashtags, emojis, twitter handles, URLs, images, videos and gifs.

This research project will investigate to what extent Twitter provide a suitable source of online reviews that can be used effectively in the generation of recommendations in a recommender system. We will explore methods of classifying review-like tweets, identifying the sentiment of the reviews and the effect of using this data in a recommender system.

The project will focus on tweets that mention or discuss hotels in the Dublin area. A collection of 2.5 million tweets posted from Dublin, collected between October 2017 and September 2018, will be used as the dataset. 

This paper will be structured as follows: Section II presents a review of the literature relating to this project. This includes studies on classification, particularly text classification, sentiment analysis, and recommender systems. Section III describes the design and methodology of the project. It outlines how the data was collected, filtered, processed, and annotated. It also describes the classification techniques used, the sentiment analysis tool used and how the sentiment scores produced were applied to the recommender system CoRE \cite{core2019}. Section IV presents an evaluation and discussion of the results of this research. Finally, Section V gives our conclusions and future works.