\begin{abstract}
This paper proposes a method of extracting review-like tweets from Twitter and incorporating the sentiment of those tweets into a recommender system. This research aims to evaluate to what extent Twitter can provide a suitable source of online reviews that can be used effectively in the generation of recommendations in a recommender system. The project focused on reviews about hotels in Dublin. We explored the use of various classification algorithms and feature representations to classify whether tweets contain reviews. The sentiment of these review-like tweets was calculated with the Stanford NLP Sentiment Analyser and used to re-rank the CoRE recommender system. The classification results were promising, showing that text classification is a valid method of extracting tweets from reviews. The best performing classifier was the Support Vector Machine. It achieved a precision score of 74\%, a recall score of 74\%, an f1-score of 73\% and an accuracy score of 74.4\%. Incorporating the sentiment score into the CoRE had the desired effect and adjusted the rankings of the hotels. However, in terms of mean percentile rank (MPR) SentiCoRE performed worse that CoRE. 
\end{abstract}