\section{Related Works}

Classification is the process of mapping observations into classes, based on some set of training data. A number of papers have investigated the performance of classification algorithms applied to Twitter data. \cite{Berm2010} investigated the performance of Support Vector Machines (SVM) and Multinomial Naïve Bayes (MNB) in classifying the sentiment of short versus long form text documents. MNB achieved better accuracy than SVM on the short form documents, from both Twitter and Blippr (a micro-review site). \cite{Rane2018} also examined the performance of classification algorithms applied to Twitter data, specifically reviews about US Airline Services. Doc2Vec feature representation was used, which involved mapping each document to a vector in space. In this study, each paragraph was mapped to a vector. They found the Random Forest Classifier performed the best.

\cite{sriram2010} evaluated the performance of different feature representations for classifying tweets into a set of generic classes: News, Opinions, Events, Deals and Private Messages. They proposed an eight-feature technique. Eight features were extracted from the tweets, one nominal (author) and seven binary (whether the tweet contained shortened words/slang, time-event phrases, opinion words, emphasis on words, currency or percentage signs, the username at the start, the username mid-tweet). They compared this technique to bag-of-words and found it performed significantly better. \cite{Berm2010} also experimented with different feature representations for Twitter data. They found that extending the unigram feature representation did not improve classification accuracy but extracting Part-of-Speech features and punctuation did. 

An Ensemble Classifier (EC) was proposed by \cite{Ankit2018} to classify the sentiment of tweets. The proposed weighted EC outperforms each of the individual base classifiers, as well as the majority voting EC. \cite{Kanakaraj2015} also found that ECs performed better in classifying the sentiment of tweets than base classifiers. They compared several base classifiers and ECs and found the ensemble methods outperformed the individual base classifiers, with Extremely Randomised Trees performing the best.

Sentiment analysis is the process of identifying the opinion expressed about a particular subject in some text. The aim is to determine whether the opinion is positive, negative or neutral, and to what extent. The two main approaches to sentiment analysis are lexicon based approaches and supervised machine learning based approaches. A lexicon-based approach works by classifying a sentence based on the number of opinion words (positive or negative words) in the sentence. Supervised machine learning approaches require labelled training data to learn a function that maps an input to an output.

\cite{Bhuta2014} reviewed different methods for the sentiment analysis of Twitter data. These included a lexicon approach and three supervised learning methods, Naive Bayes, Maximum Entropy and SVM. The three supervised learning methods outperformed the lexicon-based approach. The disadvantage of the supervised machine learning approaches is that it can be hard to know exactly what is having an effect on the algorithm and how to improve it.

The Stanford NLP Group's Sentiment Analyser \cite{stanfordSentiment2013} introduced a Recursive Neural Tensor Network (RNTN) and Sentiment Treebank. The Sentiment Treebank extended the corpus of movie reviews originally collected by Pang and Lee \cite{panglee2004}. The sentences were relabelled at a phrase level, producing a labelled parse-tree for each review. The Treebank has more finely grained sentiment labels than the original corpus, which improved how the compositional effects of sentiment in language were captured. The Sentiment Analyser achieved accuracy of 85.4\% in single sentence positive/negative classification.

Recommender systems provide suggestions for items that are most likely to be of interest to a particular user \cite{Ricci2015}. The two main methods that recommender systems use to generate recommendations are: Collaborative Filtering methods and Content-Based recommender methods. Collaborative Filtering methods analyse the behaviour of a collection of users and use this information to make recommendations based upon what users similar to the current user have liked. Content-Based (CB) recommender methods use the descriptive features of items and the preferences of individual users. CB methods recommend items similar to items the user liked or purchased in the past. Recommender systems often incorporate context information in an attempt to provide more personalised reviews. Context-aware recommender systems use contextual information (time, location, company, etc.) to improve their recommendations. The concept is that the same user could prefer different items under different conditions.

A number of hotel recommender systems have been proposed in previous literature. \cite{levi2012} proposed a cold start, context-based hotel recommender system, which uses the text of online reviews from Tripadvisor and Venere as its main data. The system asks the user to identify their trip intent (business, family, etc), nationality and preferences for certain hotel features (location, service, food, etc). Hotels are recommended based on the sentiment of reviews of users who have similar context information. The sentiment score is calculated with a lexicon based approach. They reported that users were 20\% more satisfied with their recommendations. This is a promising result for this research. The hope is that incorporating review-like tweets into the CoRE recommender will also increase user satisfaction. \cite{lin2015} also uses hotel reviews collected from TripAdvisor to recommend hotels. The system tracks the user's gestures on a mobile device to identify what part of the review the user has focused on or 'seriously read'. Feature extraction is used to extract the aspects of hotels (e.g room, food, price etc) the user considers important, and build a user interest profile. Hotels are recommended based on the user profile. The score is calculated based on the sentiment of the reviews about the aspects of the hotels the user prefers.