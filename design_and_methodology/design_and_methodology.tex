\section{Design and Methodology}

\subsection{Data Collection}
The Twitter Streaming API \footnote{\url{https://developer.twitter.com}} was used between October 2017 and September 2018 to collect a set of 2.5 million tweets. The API allows a bounding box to be specified. A bounding box was set for Dublin so all tweets collected were posted from within Dublin's bounding box.
\subsection{Data Filtering}
The raw dataset was cleaned and filtered, so that it only contained English language tweets posted from Dublin, and secondly so that it only contained tweets mentioning hotels.
\subsection{Dataset Annotation}
This part of the project involved building an annotated dataset of tweets. The tweets from the processed dataset were manually labelled as either: review-like tweets; tweets that contain some contextual information related to hotels; or irrelevant tweets. A webpage was built so that participants could annotate the tweets.
\subsection{Tweet Classification}
The annotated set of tweets was used to train multiple classifiers. Various supervised machine learning algorithms and feature representations were explored.
\subsection{Sentiment Analysis}
The Stanford NLP Sentiment Analyzer \cite{stanfordSentiment2013} was used to classify the sentiment of the classified tweets. A sentiment score was generated for each hotel.
\subsection{CoRE}
The sentiment score produced was applied to the CoRE recommender system \cite{core2019}. The effect of adding the sentiment score was analysed and evaluated.