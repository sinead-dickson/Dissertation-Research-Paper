\section{Conclusion}

In this paper a method of extracting review-like tweets from Twitter and incorporating the sentiment of those tweets into a recommender system was presented. The performance of thirteen classifiers and seven feature representation methods was evaluated and overall SVM performed best. It achieved a precision score of 74\%, a recall score of 74\%, a f1-score of 73\% and an accuracy score of 74.4\%. The best performing feature representation was unigram TF-IDF. The results confirmed that text classification is a valid method of extracting reviews from Twitter. The effect of incorperating the sentiment scores produced by the Stanford NLP Sentiment Analyser was evaluated against CoRE. The sentiment scores had the desired effect and adjusted the rankings of the hotels. However, in terms of MPR SentiCoRE performed worse than CoRE. The system was evaluated for 27 users and 21 hotels. Our results could be confirmed and would hold more weight if SentiCoRE was re-evaluated with more data.

Several promising lines of future work have been identified during the course of this research. One line of future work would be to improve the performance of the classification algorithms, by increasing the size of the training set or investigating other classification algorithms and feature representations. Another line of future work would be re-training the Stanford NLP Sentiment Analyser on a dataset more similar to our set of tweets, rather than the movie reviews it was trained on. 

